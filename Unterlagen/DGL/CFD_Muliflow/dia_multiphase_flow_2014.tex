\section{Das Modell von 2014}

In \cite{dia_2014_ijnmf} wird eine Zugkraft
(drag source) zwischen den Phasen in der Impulsgleichung hinzu
genommen, um die unterschiedliche Geschwindigkeiten der Phasen mit zu
berücksichtigen. Das System ist hyperbolisch und konservativ, was
viele numerische Ansätze zulässt. Verwendet wir das sogenannte {\it
  slope limiter centered scheme} (SLIC), welches eine Godunov-Methode
ist. Für Details wird auf das Buch von Toro verwiesen.

Es werden 2 Sätze von PDEs betrachtet, ein Satz für die gemittelte
Masse, Impuls und Energie, und ein Satz für den Volumenanteil,
Massenanteil und die relative Geschwindigkeit.

\begin{eqnarray}
&\frac{\partial}{\partial t}\rho + \frac{\partial}{\partial x} \rho u
=  0\label{eq:dia_2014_k}&\\[2mm]
&\frac{\partial}{\partial t} \rho u + \frac{\partial}{\partial x} \left(
\rho u^2 + \rho c (1-c) u_r^2 + P\right) = 0 \label{eq:dia_2014_i}&\\[2mm]
&\frac{\partial}{\partial t} \rho E + \frac{\partial}{\partial x}
\left( \rho u E + Pu + \rho c (1-c) u_r \left(uu_r + \frac{1-2c}{2}
u_r^2 + \frac{\partial e}{\partial c} \right)\right) = 0&\label{eq:dia_2014_e}
\end{eqnarray}

Die Kontinuiätsgleichung \ref{eq:dia_2014_k} entspricht Gleichung
\ref{eq:dia_2009_k} und die Impulsgleichung \ref{eq:dia_2014_i} der
Gleichung \ref{eq:dia_2009_i}. Die Energiegleichung wird
voraussichtlich nur die Summer der Eulergleichungen \ref{eq:eulerE}
für die Energie sein, nicht nachgerechnet. Dabei ist zu beachten, dass
gilt: $\rho H = \rho E+p$ und $E = e + \frac{u^2}{2} +
c(1-c)\frac{u_r^2}{2}$. Die weitere 3 Gleichungen lauten

\begin{eqnarray}
&\frac{\partial}{\partial t}\alpha\rho + \frac{\partial}{\partial x}
  \alpha\rho u = 0\label{eq:dia_2014_k1}&\\[2mm]
&\frac{\partial}{\partial t}c\rho + \frac{\partial}{\partial x}
  \left(\rho u c + \rho c (1-c) u_r \right) =
    0\label{eq:dia_2014_c1}&\\[2mm]
&\frac{\partial}{\partial t} u_r + \frac{\partial}{\partial x} (
uu_r + \frac{1-2c}{c} u_r^2 + e_c) = \pi& \label{eq:dia_2014_r}
\end{eqnarray}

Die Gleichung \ref{eq:dia_2014_k1} ist die Kontinuitätsgleichung für
eine Phase, die Gleichung \ref{eq:dia_2014_c1} verstehe ich nicht,
da sie in den anderen Gleichungen \ref{eq:dia_2014_k} und
\ref{eq:dia_2014_k1} enthaltensein müsste, da $c\rho =
\alpha_2\rho_2$, habe ich aber nicht nachgerechnet und kann dadurch
zustande kommen, dass $\alpha$ nicht konstant ist. Die Gleichung
\ref{eq:dia_2014_r} entspricht der Gleichung \ref{eq:dia_2009_r}.  Die
Kraft zwischen an der Grenzschicht der Phasen $\pi$ hat die einfache
Form
\[
\pi = \kappa (u_2-u_1) = \kappa u_r
\]

Anstatt z.B. Geschwindigkeit oder Dichte kann in den allgemeinen
Gleichungen auch die Entropie eingesetzt werden und damit kommt man
analog zur den anderen Gleichungen auf eine Differentialgleichung für
die Entropie, die sich z.B. über $Tds = dh - v dp$ oder $T ds = de - p
dv$ in die Gleichung für die Energie bzw. Enthalpie umrechnen lässt.

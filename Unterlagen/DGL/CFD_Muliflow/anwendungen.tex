\section{Andere Forschungsgruppen, Anwendungen und Förderprogramme}

Selbstverständlich ist die Gruppe um Toro weiterhin auf diesem Gebiet
aktiv, \\
\verb+http://eleuteriotoro.com/research/+
, der an der University of Trento arbeitet.





Aktivitäten gibt es an der RWTH Aachen:\\
\verb+http://www.igpm.rwth-aachen.de/node/200+ An Institut für
Geometrie und praktische Mathematik wird an dem Projekt {\it
  ``Dynamics of cavitation bubbles in compressible two-phase fluid
  flow''} gearbeitet.


Frei verfügbaren Code in C++ steht unter\\
\verb+http://amroc.sourceforge.net/examples/index.htm+ für die Euler
Gleichungen. Da gibt es bestimmt noch mehr.

Ein weiteres interessantes Thema könnte die Parallelisierung sein. Das
wird z.B. von einer Gruppe in Cambridge betrieben\\
\verb+http://www.many-core.group.cam.ac.uk/projects/blakely.shtml+, die
eine Seite zu ``Riemann-problem-based methods on the GPU'' haben.

Ein weitere Platz ist das ICCS - International Center for Computationl
Science, darin die Gruppe Infrastructure for Astrophysics Applicaitons
Computing,\\
\verb+http://iccs.lbl.gov/research/isaac/GAMER_Framework.html+, die
ihre Code bereits hybrid in OpenMP/MPI/GPU parallelisiert haben.


Förderprogramme sind schwer zu finden. Leider gerade verpasst haben wir\\
\verb+http://www.bmbf.de/foerderungen/25515.php+ bzw. \\
\verb+http://www.eranetmed.eu/+

Dann gäbe es noch das EU-Programm\\
\verb+http://ec.europa.eu/programmes/horizon2020/+.\\
Vielleicht lässt sich darüber was finden, 

Ein Herr Eric Goncalves Da Silva vom ENSMA,\\
\verb+http://www.ensma.fr/+\\
wäre im Falle eines EU-Projekts an der Mitarbeit interessiert.


Zusammenstellung von Peter an Anwendungen aus den entsprechenden
WWW-Seiten:

\begin{itemize}
\item Toro

My own experience is related to compressible, reactive multiphase
flows in propulsion technology, in which complex moving boundaries are
present. Past and current work in this area has been funded by the
British Ministry of Defense via DERA.

Another area of application of my interest is in Nuclear Reactor
Safety and Design. Dia war PhD Student bei diesem Projekt.

THE REMISSION PROJECT REMISSION: a long-term research project on
Research into Mathematical modelling of Multiple Sclerosis and its
vascular connection Computer simulation of blood flow in the
intra/extra cranial venous system in humans with multiple sclerosis
and the CCSVI condition. This research programme is motivated by the
recently proposed association between multiple sclerosis (MS) and a
vascular anomaly termed chronic cerebro-spinal venous insufficiency
(CCSVI) by Zamboni and collaborators. The CCSVI condition is
characterized by the presence of obstructions of various kinds in the
extracranial veins. Such obstructions prevent a normal drainage of
blood from the brain to the heart. CCSVI is present in a relevant
number of MS patients and such occurrence is of great clinical
interest.

\item INSTITUT FÜR GEOMETRIE UND PRAKTISCHE MATHEMATIK der RWTH Aachen

The primary objective of the project is to provide an accurate
description and prediction of all wave and flow phenomena arising in
the dynamical two--phase system formed by a collapsing cavitation
bubble. Cavitation processes in liquids, i.e., bubble formation and
collapse of the bubble, play a significant role in numerous technical
applications such as fluid transport, tube and nozzle flow and water
turbines. By means of cavitation, shock and expansion waves may occur
as well as liquid jets with high kinetic energy. At a solid wall these
effects may cause material damage by erosion.

\item Gruppe in Cambridge

The Rayleigh–Taylor instability, or RT instability (after Lord
Rayleigh and G. I. Taylor), is an instability of an interface between
two fluids of differentdensities which occurs when the lighter fluid
is pushing the heavier fluid.[1][2] Examples include supernova
explosions in which expanding core gas is accelerated into denser
shell gas,[3][4] instabilities in plasma fusion reactors,[5] and the
common terrestrial example of a denser fluid such as water suspended
above a lighter fluid such as oil in the Earth's gravitational field.

\item ICCS - International Center for Computationl Science, darin die Gruppe
 Infrastructure for Astrophysics Applicaitons Computing

GAMER is a GPU-accelerated Adaptive MEsh Refinement Code for
astrophysical applications.  The Kelvin–Helmholtz instability (after
Lord Kelvin and Hermann von Helmholtz) can occur when there is
velocity shear in a single continuous fluid, or where there is a
velocity difference across the interface between two fluids. An
example is wind blowing over water: The instability manifests in waves
on the water surface. More generally, clouds, the ocean, Saturn's
bands,Jupiter's Red Spot, and the sun's corona show this instability.

\end{itemize}




%%% Local Variables: 
%%% mode: latex
%%% TeX-master: "cfd_multiflow"
%%% End: 

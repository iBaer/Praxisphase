\documentclass{article}
\usepackage{epsfig}
\usepackage{latexsym}
\usepackage{graphicx}
\usepackage[usenames]{color}
\usepackage{german}
\usepackage[utf8]{inputenc}

\textheight=24.5 true cm
\textwidth=15.4 true cm
%\voffset=-1.0 true cm
%\hoffset=-0.5 true cm
\topmargin 0pt
\headheight 0pt
\headsep 0pt
\hfuzz 0.3 cm
\oddsidemargin 0cm
\evensidemargin 0cm
\parindent 0pt
\parskip 10pt

\newcommand{\bes}{\begin{eqnarray}}
\newcommand{\ees}{\end{eqnarray}}
\newcommand{\nn}{\nonumber }
\newcommand{\nnn}{\nonumber \\}
%\renewcommand{\thesection}{\arabic{section}.}
\renewcommand{\theequation}{\thesection.\arabic{equation}}
\renewcommand{\d}{\;\nabla \cdot }
%\renewcommand{\d}{\;\mbox{div}\;}
\newcommand{\g}{\;\mbox{grad}\;}
\newcommand{\mbf}{\mathbf}
\definecolor{dred}{rgb}{.6,0,0}
\newcommand{\drd} {\color{dred}}

%========================================================================
\begin{document} 
%========================================================================
%\tableofcontents

\section{Current work in 1-d}
We are looking at the following 3 equations in 1 dimensions of you
publications:
\begin{eqnarray}
\frac{\partial}{\partial t}\rho + \frac{\partial}{\partial x} \rho u
&=&  0\label{eq:dia_2009_k}\\
\frac{\partial}{\partial t} \rho u + \frac{\partial}{\partial x} (
\rho u^2 + \rho c (1-c) u_r^2 + P)&=& 0 \label{eq:dia_2009_i}\\
\frac{\partial}{\partial t} u_r + \frac{\partial}{\partial x} (
uu_r + \frac{1-2c}{2} u_r^2 + \Psi(P)) &=& 0 \label{eq:dia_2009_r}
\end{eqnarray}
With the variables
\[
U = \left[\begin{array}{c}u_1 \\ u_2 \\ u_3\end{array}\right]
=  \left[\begin{array}{c}\rho \\ \rho u \\ u_r\end{array}\right]
\]
the Jacobian is
\begin{equation}
J = \left(\begin{array}{ccc} 0 & 1 & 0\\ - \frac{u_2^2}{u_1^2} + c
  (1-c) u_3^2 + \partial_{u_1} P & \frac{2 u_2}{u_1} & u_1 c (1-c) 2
  u_3 \\ \\ - \frac{u_2}{u_1^2} u_3 + \partial_{u_1} \Psi(\rho) &
  \frac{u_3}{u_1} & \frac{u_2}{u_1} + (1-2c) u_3
\end{array}\right)\label{eq:jacobi_general}
\end{equation}

Now the question we are still looking at is, what EOS to use and what
to keeps constant. Currently we are looking at 3 cases

\begin{enumerate}
\item As in you publication and in the Fortran program:
  \begin{itemize}
  \item Use the EOS
    \begin{equation}
      P = K_2 \rho_2^\gamma 
    \end{equation}
  \item Keep $c=\alpha \rho_2/\rho$ constant (In Fortran: \verb+CCL+)
  \item Do the initialization step as in the Fortran program, compute
    $K_\rho$ (In Fortran: \verb+CT = PLL/(RHOL**G)+) for a second EOS
    \begin{equation}
      P =  K_\rho \rho^{\gamma}
    \end{equation}
    and keep $K_\rho$ constant
  \item In the equation for $\rho u$ use
    \[
    P =  K_\rho \rho^{\gamma}
    \]
  \item In the equation for $u_r$ use
    \[
    \Psi = \frac{\gamma}{\gamma-1} K_2^{1/\gamma}
    P^{(\gamma-1)/\gamma} - \frac{P}{\rho_1}
    \]
  \item The corresponding Jacobian would be

    \begin{equation}
      J = \left(\begin{array}{ccc}
        0 & 1 & 0\\[3mm]
        - \frac{u_2^2}{u_1^2} + c (1-c) u_3^2  + \gamma K_\rho u_1^{\gamma-1}
        & \frac{2 u_2}{u_1}  &  u_1 c (1-c) 2 u_3  \\[3mm]
        - \frac{u_2}{u_1^2} u_3 +
        \left(\frac{K_2}{K_\rho}\right)^{1/\gamma}
        \gamma K_\rho u_1^{\gamma-2} - \frac{\gamma K_\rho u_1^{\gamma-1}}{\rho_1} 
        &   \frac{u_3}{u_1}  &  
        \frac{u_2}{u_1} + (1-2c) u_3 
      \end{array}\right)\label{eq:jacobi_v1}
    \end{equation}   
    
  \end{itemize}
  
\item In the second version we don't use the second EOS $P = K_\rho
  \rho^{\gamma}$ at all. The set of equations is the same and we
  initialize all constants in the same way, but in each step
  \begin{itemize}
  \item we are computing
    \[
    \rho_2 = \frac{c\rho \rho_1}{\rho (c-1) + \rho_1}
    \]
  \item and with this density value for $\rho_2$
    \[
    P = K_2 \rho_2^\gamma.
    \]
  \item We are using these values for $\rho_2$ and $P$ for the
    equations and the CLF condition.
  \item This makes the derivation of the
    Jacobian much more complicated and we get
  \begin{equation}
    J = \left(\begin{array}{ccc}
      0 & 1 & 0\\[3mm]
      - \frac{u_2^2}{u_1^2} + c (1-c) u_3^2  + 
      K_2 \gamma \rho_2^{\gamma-1} \frac{c\rho_1^2}
      {(\rho_1 + (c - 1)\rho)^2}
      & \frac{2 u_2}{u_1}  &  u_1 c (1-c) 2 u_3  \\[3mm]
      - \frac{u_2}{u_1^2} u_3 + K_2 
      \left(\gamma\rho_2^{\gamma-2} -
      \frac{\gamma\rho_2^{\gamma-1}}{\rho_1}\right)\frac{c\rho_1^2}
           {(\rho_1 + (c - 1)\rho)^2}
           &   \frac{u_3}{u_1}  &  
           \frac{u_2}{u_1} + (1-2c) u_3 
    \end{array}\right)\label{eq:jacobi_v2}
  \end{equation}
  still wiht $\rho_2$ in it.
  \end{itemize}
  
\item The last alternative is to look just at the EOS for the whole
  density $P = K_\rho \rho^{\gamma}$.
  \begin{itemize}
    \item In this case we compute $K_\rho$ as in the Fortran program
      to compare the results.
    \item In the equation for $u_r$ use again this equation for $P$,
      but we have to recompute $\Psi$.
    \item Using $P = K_\rho \rho^{\gamma}$ I get for $\Psi$
      \begin{eqnarray}
        \Psi &=& \frac{1}{c}\left(\frac{\gamma}{\gamma-1} \frac{P}{\rho} -
        \frac{P}{\rho_1}\right)\\
        &=& \frac{K_g}{c}\left(\frac{\gamma}{\gamma-1} \rho^{\gamma-1} -
        \frac{ \rho^{\gamma}}{\rho_1}\right)\label{eq:ur_Psi_2_rho}
      \end{eqnarray}
    \item With these expressions I get for the Jacobian
      \begin{equation}
        J = \left(\begin{array}{ccc}
          0 & 1 & 0\\
          - \frac{u_2^2}{u_1^2} + c (1-c) u_3^2  + K_\rho \gamma u_1^{\gamma-1}
          & \frac{2 u_2}{u_1}  &  
          u_1 c (1-c) 2 u_3  \\
          \\
          - \frac{u_2}{u_1^2} u_3 +
          \frac{1}{c}\left(\frac{1}{u_1}-\frac{1}{\rho_1} \right)
          K_\rho \gamma u_1^{\gamma-1}
          &   \frac{u_3}{u_1}  &  
          \frac{u_2}{u_1} + (1-2c) u_3 
        \end{array}\right)\label{eq:jacobi_v3}
      \end{equation}
  \end{itemize}
  
\end{enumerate}

Currently we run all three cases with Lax-Friedrich and FORCE, but the
code has to be checked again.

\section{Current work in 2-d}

We implemented the 5 equations as discussed before, but there are
still errors in the code. Next week I want to have a look at the
Jacobian and to help to find the bugs in the code.

%========================================================================
\end{document}
%========================================================================

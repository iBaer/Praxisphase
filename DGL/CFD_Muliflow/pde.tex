\section{Partielle Differentialgleichungen - PDFs }\label{sec:pdg}

Hier möchte ich nur das Teil von dem allgemeinen Thema PDFs
vorstellen, der für 2-Phasenströmungen gebraucht wird.  

Betrachten wir Funktionen von 2 Variablen, einer Raumrichtung $x$ und
der Zeit $t$.  Die implizite Form einer partiellen
Differentialgleichung für eine Funktion u, die von zwei Variablen $x$
und $t$ abhängt, lautet allgemein
\begin{equation}
    F\left(x,t,u(x,t),\frac{\partial u(x,t)}{\partial
      x},\frac{\partial u(x,t)}{\partial t}, \ldots,\frac{\partial^2
      u(x,t)}{\partial x \partial t},\ldots \right) = 0,
\end{equation}
wobei F eine beliebige Funktion ist.

Wikipedia: Eine sehr einfache partielle Differentialgleichung ist die
lineare Transportgleichung in einer Raumdimension. Sie hat die Form
\begin{equation}
\frac{\partial u(x,t)}{\partial t} + c \frac{\partial u(x,t)}{\partial
  x} = 0
\end{equation}

mit einem konstanten reellen Parameter $c$. Die gesuchte Funktion $u(x,t)$
ist von zwei Variablen abhängig, wobei üblicherweise $x$ den Ort und $t$
die Zeit bezeichnet. Nehmen wir an, dass die Funktion u zu einer
gewissen Zeit (etwa zu der Zeit $t=0$) bekannt ist. Es gelte also für
alle x im Definitionsbereich von u eine Beziehung der Form
$u(x,0)=g(x)$, wobei $g$ eine beliebig vorgegebene, mindestens einmal
differenzierbare Funktion sei (Anfangsbedingung). Dann ist für
beliebige Zeiten $t$ die Lösung der linearen Transportgleichung durch
$u(x,t) = g(x-ct)$ gegeben. Diese Gleichung bedeutet nichts anderes,
als dass die Anfangsdaten $g$ in unveränderter Form mit der
Geschwindigkeit $c$ in Richtung der positiven $x$-Achse verschoben
(„transportiert“) werden (längs der Charakteristik der Gleichung),
siehe nebenstehendes Bild. Ein Anwendungsbeispiel wäre der Transport
eines im Wasser gelösten Stoffes mit der Strömung des Wassers, also
zum Beispiel der Transport von Schadstoffen in einem Fluss (wobei die
Diffusion des Stoffes vernachlässigt wird).

Bei linearen PDGs kommt die Funktion bzw. deren Ableitungen in jedem
Summanden nur linear vor, Strömungsprobleme sind jedoch im Normalfall
nichtlinear. Weiterhin haben wir mehrere und nicht nur eine PDG.
Tauchen nur Ableitungen erster Ordnung auf, heißen die Gleichungen
entsprechend Erste-Ordnung Partielle Differentialgleichungen. Eine
Verallgemeinerung der lineare Transportgleichung besteht
dementsprechend darin, dass $u$ nicht eine Funktion, sondern ein
Vektor von unbekannten Funktionen $\vec{u} = \vec{u} =
(u_1,\ldots,u_l)$ ist, und der Anteil mit dem konstanten reellen
Parameter $c \frac{\partial u(x,t)}{\partial x}$ ebenfalls ein Vektor
  von Funktionen ist.
\begin{equation}
\frac{\partial \vec u}{\partial t} + 
\frac{\partial}{\partial x} \vec {f} (\vec u) = 0,
\end{equation}
Betrachten wir jetzt die Kettenregel für eine Funktion einer Funktion
\begin{equation}
\frac{\partial \vec f(u(x))}{\partial x} = \frac{\partial\vec
  {f}(u(x)) }{\partial u} \frac{\partial u}{\partial x},\nonumber
\end{equation}
so lautet die Verallgemeinerung bei einer Funktion, die von über viele
Funktionen von einer Variablen abhängt
\begin{equation}
\frac{\partial \vec {f} (u_1(x),\ldots u_s(x))}{\partial x} = \sum_i^s
\frac{\partial\vec {f} }{\partial u_i} \frac{\partial u_i}{\partial
  x},
\end{equation}
Für nur eine PDG mit $u_j$ folgt damit
\begin{equation}
\frac{\partial u_j}{\partial t} + \sum_{i=1}
\frac{\partial f_j }{\partial u_i} \frac{\partial u_i}{\partial
  x} = 0.
\end{equation}
oder in Vektorschreibweise für alle Gleichungen
\begin{equation}
\frac{\partial \vec u}{\partial t} + A 
\frac{\partial \vec u}{\partial x} = 0.
\end{equation}
mit der $s \times s$ Jocobi Matrix
\begin{equation}
A =
\left(\begin{array}  {ccc}
\frac{\partial f_1}{\partial u_1} & \cdots &
  \frac{\partial f_1}{\partial u_s} \\ \vdots & \ddots & \vdots
  \\ \frac{\partial f_s}{\partial u_1} & \cdots & \frac{\partial
    f_s}{\partial u_s} 
\end{array}\right)
\end{equation}
Das Gleichungssystem wird als hyperbolisch bezeichnet, wenn die Matrix
$A$ diagonalisierbar ist und alle Eigenwerte reell sind.

Ein einfaches Beispiel der sogenannten linearisierten Gasdynamik aus
\cite{toro}. Die partiellen Ableitungen werden der Einfachheit halber
abgekürzt zu $\frac{\partial f}{\partial x} = \partial_x f = f_x $.
\begin{eqnarray}
&\partial_t \rho     + \partial_x (\rho u) = 0&\\[2mm]
&\partial_t (\rho u) + \partial_x (\rho u^2 + \rho a^2) = 0&
\end{eqnarray}
oder in Vektorschreibweise
\begin{eqnarray}
&\partial_t \vec{u} +\partial_x \vec{f}(\vec{u}) = 0&\\ &\vec{u} =
  \left[\begin{array}{c}\rho \\ \rho u\end{array}\right],\quad \vec{f}
  = \left[\begin{array}{c}\rho u \\[2mm] \rho u^2 + \rho
      a^2 \end{array}\right]. &
\end{eqnarray}
Diese beiden Gleichungen können mit der Produktregel aufgelöst werden
in jeweils eine Gleichung für $\rho$ und eine für $u$ zu
\begin{eqnarray}
\partial_t \rho + \rho \partial_x u + u \partial_x \rho &=&
0 \label{2dconti}\\[2mm] 
\rho \partial_t u + u \partial_t \rho + 2 u \rho \partial_x u
+ u^2 \partial_x \rho + a^2 \partial_x \rho  &=& 0 \nonumber 
\end{eqnarray}
bzw. für die 2. Gleichung
\begin{eqnarray}
\partial_t u + u \partial_x u + \frac{1}{\rho} a^2
\partial_x \rho &=& 0 \label{2dmon}
\end{eqnarray}
Zur Herleitung von \ref{2dmon} wurde \ref{2dconti} verwendet und die
Gleichung durch $\rho$ geteilt. Das kann wiederum in
Matrixschreibweise umgeformt werden:
\begin{eqnarray}
&\partial_t \vec{u} + A\;\; \partial_x \vec{u} = 0& \label{2dvec}\\ &\vec{u} =
  \left[\begin{array}{c}\rho \\ u\end{array}\right],\quad A =
  \left(\begin{array}{cc} u & \rho \\[2mm] \frac{1}{\rho} a^2 &
    u\end{array}\right). &
\end{eqnarray}
Falls die Matrix $A$ diagonalisierbar ist, existiert eine Matrix $K$ mit
\begin{equation}
A = K \Lambda K^{-1} \quad oder \quad \Lambda = K^{-1} A K,
\end{equation}
wobei $\Lambda$ eine Diagonalmatrix ist mit den Eigenwerten auf der
Diagonalen. Sind alle Eigenwerte reell, handelt es sich um ein
hyperbolisches Gleichungssystem. Wird nun Gleichung \ref{2dvec} von
links mit $K^{-1}$ multipliziert, zerfällt das Gleichungssystem in 2
unabhängige Gleichungen.
\begin{eqnarray}
&K^{-1}\partial_t\vec{u} = \vec{w}_t,\quad K^{-1}\partial_x\vec{u} = \vec{w}_x
\quad K^{-1} A\;\; \partial_x \vec{u} = K^{-1}A\; K K^{-1}\;\partial_x \vec{u}
& \\
&\vec{w}_t + \Lambda \vec{w}_x = 0 &\nonumber  
\end{eqnarray}
Hängen die Eigenwerte nicht von $\rho$ und $u$ ab, ergeben sich 2
Transportgleichungen und die Eigenwerte entsprechen wie oben den
Geschwindigkeiten. Der größte Eigenwert bestimmt also die
Geschwindigkeit im System.  Diese Betrachtungen sind deshalb wichtig,
da die Geschwindigkeiten, also die Eigenwerte bei numerischen
Verfahren das Verhältnis der Schrittweiten in Raum und Zeit bestimmen,
um physikalische Ergebnisse zu bekommen.

Wikipedia: Die Courant-Friedrichs-Lewy-Zahl (CFL-Zahl oder auch
Courant-Zahl) wird für die Diskretisierung zeitabhängiger partieller
Differentialgleichungen verwendet. Sie gibt an, um wie viele ``Zellen''
sich eine betrachtete Größe pro Zeitschritt maximal fortbewegt:
\begin{equation}
  c = \frac{v \cdot \Delta t}{\Delta x} \label{eq:cfl}
\end{equation}

Dabei ist $c$ die Courant-Zahl, $v$ die Geschwindigkeit, in unserem
Fall der größte Eigenwert, $\Delta t$ der diskrete Zeitschritt und
$\Delta x$ der diskrete Ortsschritt. Motiviert wird dies durch die
CFL-Bedingung, die aussagt, dass das explizite Euler-Verfahren nur für
$c<1$ stabil sein kann. Ähnliche Bedingungen gelten auch für andere
Diskretisierungsschemata. Das heißt für uns mit
\begin{equation}
\Delta t \leq \frac{\Delta x}{v_{max}}
\end{equation}
In Normalfall hängen die Eigenwerte jedoch von $\rho$ und $u$, d.h. in
jedem Zeitschritt müssen die Eigenwerte für alle Werte von$\rho$ und
$u$ neu berechnet werden. Der größte Eigenwert ergibt sich jedoch
meist bei dem maximalen Wert von $u$, so dass nur diese maximale Wert
im Raum bestimmt werden muss. Mit diesem Wert und dem zugeörigen
anderen Werte an diesem Ortpunkt können dann die Eigenwerte der Matrix
berechnet werden müssen.










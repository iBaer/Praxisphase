\begin{center}
{\large \bf Zusammenarbeit mit Dia Zeidan}
\end{center}

In den Arbeiten von Dia wird das sogenannte Riemann-Problem für
Mehrphasenströmungen behandelt und als mögliche Anwendung Aktivitäten
von Vulkanen genannt. Zum Verständnis des Problem, hier ein paar
Erläuterungen, insbesondere zu den verwendeten Gleichungen.


\section{Einleitung}


Wikipedia: Als Riemann-Problem (nach Bernhard Riemann) wird in der
Analysis ein spezielles Anfangswertproblem bezeichnet, bei dem die
Anfangsdaten als konstant definiert werden, bis auf einen Punkt, in
welchem sie unstetig sind. Riemann-Probleme sind sehr hilfreich für
das Verständnis hyperbolischer partieller Differentialgleichungen, da
in ihnen alle Phänomene wie Schocks, Verdichtungsstöße oder
Verdünnungswellen auftauchen. Ebenfalls sind auch für komplizierte
nichtlineare Gleichungen wie die Euler-Gleichungen exakte Lösungen
konstruierbar, was nicht für beliebige Anfangsdaten möglich ist.

Allgemein werden durch hyperbolische Gleichung Wellen und deren
Ausbreitung beschrieben. Partielle Differentialgleichungen erster
Ordnung sind hyperbolisch, falls die zugehörige Jacobi-Determinante
nur reelle Eigenwerte hat. Im Unterschied zu
parabolischen und elliptischen Gleichungen werden Lösungen
hyperbolischer Gleichungen wenig bis gar nicht gedämpft. Das führt
einerseits zu einer komplizierten Lösungstheorie, da mit weniger
Differenzierbarkeit gerechnet werden kann. Andererseits können sich
Wellen erst durch diese fehlende Dämpfung über weite Strecken
ausbreiten.

Oft genannt werden die Euler-Gleichungen: Die Euler-Gleichungen oder
auch eulersche Gleichungen (nach Leonhard Euler) sind ein
mathematisches Modell zur Beschreibung der Strömung von reibungsfreien
Fluiden. Es handelt sich um ein partielles
Differentialgleichungssystem 1. Ordnung, das sich als Sonderfall der
Navier-Stokes-Gleichungen ergibt, falls die innere Reibung
(Viskosität) und die Wärmeleitung des Fluids vernachlässigt werden.

Eine weitere Eigentschaft der untersuchten Systeme ist, dass sie
konservativ sind. Das bedeute physikalisch, dass sich die Kraft durch
ein Potential herleiten lässt, oder dass die totale mechanische
Energie erhalten bleibt. Wir werde es also im wesentlichen mit 3
Erhaltungssätzen, formuliert als partielle Differentialgleichungen zu
tun habe, Massenerhaltung, Impulserhaltung und Energieerhaltung, die
numerisch gelöst werden müssen.

In der numerischen Mathematik tauchen Riemann-Probleme in natürlicher
Weise in Finite-Volumen-Verfahren zur Lösung von Erhaltungsgleichungen
auf. Dort werden die Riemann-Probleme approximativ mittels so
genannter Riemann-Löser angegangen.

Zum weiteren Verständnis werden alle oben genannten Begriffe
ausführlich eingeführt. Einigen aus dem Bereich der Strömungsmechanik
wie die Navier-Stokes und die Euler-Gleichungen werden
aufgeführt. Dann werde ich auf 2-Phasenströmungen und numerische
Methoden eingehen.


\section{Grundgleichungen der Strömungsmechanik}

{\bf Gleichungen} 

Grundlage dieser Einführung ist das Buch von Ferziger und Peric
\cite{cfd}. Für Materie oder {\it control mass} (CM) gelten die
Grundgleichungen
\begin{equation}
\frac{dm}{dt} = 0 \qquad \frac{dm \vec{v}}{dt} = \sum \vec{F}.
\end{equation}
Zur Beschreibung von Fluiden muss der Übergang von Massen zu Volumen
({\it control volume} CV) vollzogen werden.
Für eine intensive erhaltene Größe $\phi$ pro Einheitsmasse gilt
\begin{equation}
\Phi = \int_{\Omega_{CM}} \rho \phi d \Omega,
\end{equation} 
wobei $\Omega_{CM}$ für das von CM eingenommene Volumen steht. Es gilt
\begin{equation}
\frac{d}{dt} \int_{\Omega_{CM}} \rho \phi d \Omega 
=
\frac{d}{dt} \int_{\Omega_{CV}} \rho \phi d \Omega 
+
\int_{S_{CV}} \rho \phi (\vec{v}-\vec{v}_b)\cdot \vec{n} d S.\label{eq:cm_cv}
\end{equation} 
$\vec{v}_b$ ist Strömungsgeschwindigkeit der Oberfläche und kann im folgenden
$= 0$ gesetzt werden. Weiterhin werden keine Quellen berücksichtigt.

{\bf Massengleichung}

Setze $\phi=1$.
\begin{equation}
\frac{\partial}{\partial t} \int_{\Omega_{CV}} \rho d \Omega
+
\int_{S_{CV}} \rho (\vec{v} \cdot \vec{n}) d S
= 0.
\end{equation}
In differentieller Form ergibt sich mit dem Gaußschen Satz die
Kontinuitätsgleichung
\begin{equation}
\frac{\partial \rho }{\partial t} + \d (\rho \vec{v}) =
0.\label{eq:masse_diff}
\end{equation}


{\bf Impulsgleichung}

Setze $\phi=\vec{v}$. Dann gilt für ein festes Volumen
\begin{equation}
\frac{\partial}{\partial t} \int_{\Omega} \rho \vec{v} d \Omega +
\int_{S} \rho \vec{v} \; (\vec{v}\cdot \vec{n}) d S = \sum \vec{F}
\end{equation} 

Der Kraftterm setzt sich aus 2 Anteilen zusammen:
\begin{itemize}
\item Oberflächenkräfte: Druck (pressure), Normal- und Scherspannung (normal
  and shear stresses), Oberflächenspannung (surface tension) etc.;
\item Körperkräfte: Gravitation, Zentrifugal- und Corioliskraft,
  elektromagnetische Kraft etc..
\end{itemize}

Für sogenannte Newtonsche Strömungen können die Oberflächenkräfte durch den
folgenden Spannungstensor $T$ beschrieben
werden
\begin{eqnarray*}
T &=& S - p I \\
S &=& \mu \left( 2 D - \frac{2}{3} (\d \vec{v})\; I \right) .
\end{eqnarray*}
Dabei ist $\mu$ die dynamische Viskosität, $I$ der Einheitstensor und
$p$ der statische Druck.  Bei Ferziger und Peric und bei Wikipedia
wird $T$ als Spannungstensor bezeichnet, während im Fluent-Handbuch
$S$ als Spannungstensor bezeichnet wird, also ohne den Term
proportional zu $p$, sondern nur der Term proportional zur
Viskosität. Der Tensor $D$ (strain deformation tensor) ergibt sich zu
\begin{eqnarray*}
D &=&\frac{1}{2}\left[ (\nabla \otimes \vec{v})^T + (\nabla \otimes \vec{v})\right]\\
D_{ij} &=& \frac{1}{2}\left(\frac{\partial v_i}{\partial x_j} 
                         +\frac{\partial v_j}{\partial x_i} \right).
\end{eqnarray*}
Im Falle von reibungsfreien Fluide ist $\mu=0$. Mit der Körperkraft
pro Einheitsmasse $f$ lautet nun die Impulsgleichung
\begin{equation}
\frac{\partial}{\partial t} \int_{\Omega} \rho \vec{v} d \Omega 
+
\int_{S} \rho \vec{v} \; (\vec{v}\cdot n) d S 
= 
\int_{S} (T \cdot \vec{n}) d S 
+
\int_{\Omega} \rho \vec{f} d \Omega
\end{equation} 

Damit lautet die differentielle Form der Impulsgleichung, wieder über
den Gaußschen Satz, wobei bei der Umwandlung des Oberflächenintetrals
in ein Volumenintegral der Term $\vec{v} \; (\vec{v}\cdot n)$ in $v$
zu einem dyadischen Produkt wird, also zu $\vec{v}\vec{v}^T =
\vec{v}\otimes\vec{v}$

\begin{equation}
\frac{\partial (\rho \vec{v})}{\partial t} + \d (\rho \vec{v}\otimes
\vec{v}) = - \nabla p + \d S + \rho \vec{f}\label{eq:impuls_diff}
\end{equation}

Fluent bietet verschiedene Optionen für nicht-Newton\-sche Strömungen mit
verschiedenen Gesetzmäßigkeiten für die Viskosität an, auf die im Ferziger
nicht weiter eingegangen wird.  Die Kontinuitätsgleichung und die
Impulsgleichung zusammen heißen Navier-Stokes Gleichungen.  Daneben können
weitere Gleichungen aufgestellt werden, wobei z.B. als erhaltene Größe $\phi$
die Energie eingesetzt wird.


{\bf Energiegleichung}

Wird in Gleichung \ref{eq:cm_cv} für $\phi$ die Energie eingesetzt,
folgt (ohne Begründung und nach Wikipedia)
\begin{equation}
\frac{\partial (\rho E)}{\partial t} + \d (\rho \vec{v} H) = 
\d \left(S \vec{v} - \vec{W} \right) + q - \rho
\vec{v}\vec{g} \label{eq:transport_E}
\end{equation}
mit der Enthalpie pro Einheitsmasse 
\[
H = E+p/\rho,
\] dem Wärmefluss
\[
\vec{W} = - \kappa \nabla T
\]
mit dem Wärmeleitkoeffizienten $\kappa$, dem Anteils $s_{i,j}$ des
Spannungstensors und einem allgemeinen Quellterm $q$. Die totale
Energie pro Einheitsmasse ist die Summe aus der inneren Energie $e$,
der kinetischen und der potentiellen Energie
\[
E = e + \frac{1}{2} |\vec{v}^2| - h |g|
\]

Der Druck wird über die Zustandsgleichung
\[
p = (\gamma-1) \rho \left(E -  \frac{1}{2} |\vec{v}^2|  - h |g|\right)
  = (\gamma-1) \rho e
\]
mit dem adiabatischem Koeffizienten 
\[
\gamma = \frac{c_p}{c_v}
\]
berechnet. Mit der Anzahl der Freiheitsgrade $f = f_{trans} + f_{rot}
+ f_{vib}$ ergibt sich in guter Näherung
\[
\gamma = \frac{f+2}{f}.
\]

Ein einfache Gleichung erhält man über die ideale Gasgleichung. Dann gilt
\[
\rho R T = p \quad \mbox{und} \quad e = c_v T = \frac{RT}{\gamma-1} =
\frac{p}{\rho (\gamma-1)}
\]

Der Druck ist also unabhängig von der Anzahl der Freiheitsgrade, da
die innere Energie gerade den Faktor $\gamma-1$ auffängt.

%\newpage

{\bf Erhaltungsgleichung skalarer Größen}

Analog zu Gleichung (\ref{eq:cm_cv}) lässt sich eine Erhaltungsgleichung für
eine allgemeine skalare Größe $\phi$ aufstellen
\begin{equation}
\frac{d}{dt} \int_{\Omega_{CM}} \rho \phi \; d \Omega = \frac{d}{dt}
\int_{\Omega_{CV}} \rho \phi \; d \Omega + \int_{S_{CV}} \rho \phi \;
\vec{v}\cdot \vec{n} \; d S  = \sum f_\phi
\end{equation} 
wobei die rechte Seite jetzt nicht ein Kraftterm ist wie für die
Impulsgleichung sondern den Transport der Größe $\phi$ durch Diffusion
(Übertragung, Transport durch Werteunterschiede wie z.B.
Temperaturunterschiede) aus dem Volumen oder eine Quelle bzw. Senke
beschreibt. Generell ist ein Diffusionsterm (Ausgleich bei unterschiedlichen
Werten) zu berücksichtigen, der durch das {\it Fourier Gesetz} (Wärmeleitung)
für die Wärme und durch das {\it Ficksche Gesetz} (Transport durch
Konzentrationsunterschiede) für die Masse als Gradiententerm genähert werden
kann.
\begin{equation}
f_\phi^d = \int_S \Gamma (\g \phi \cdot n)  \; d S
\end{equation} 
$\Gamma$ ist der Diffusionskoeffizient der Größe $\phi$. Ein Beispiel für eine
Transportgleichungen erhält man z.B. beim Betrachten der Energie bzw.
Temperatur. Für unser Problem werden jedoch keine weiteren Gleichungen
benötigt.

Für allgemeine Betrachtungen numerischer Methoden wird mit diesem Ansatz von
folgender allgemeinen Form einer Erhaltungsgleichung ausgegangen
\begin{equation}
\frac{d}{dt} \int_{\Omega_{CV}} \rho \phi d \Omega 
+
\int_{S_{CV}} \rho \phi \; \vec{v}\cdot \vec{n} d S.
= 
\int_S \Gamma (\g \phi \cdot n)  \; d S
+
\int_{\Omega_{CV}} q_\phi d \Omega \label{eq:transport_int}
\end{equation} 
wobei $q_\phi$ eine Quelle bzw. Senke der Größe $\phi$ ist.
In differentieller Form lautet die Gleichung
\begin{equation}
\frac{\partial (\rho \phi)}{\partial t} + \d (\rho \phi \vec{v}) =
\d (\Gamma \g \phi)  + q_\phi \label{eq:transport_dif}
\end{equation}


Analysen der Gleichungen werden meist in dimensionsloser Form
durchgeführt. Dabei werden die auftretenden Größen mit Referenzgrößen skaliert.
Es treten 3 die Strömung bestimmende Konstanten auf:
\begin{eqnarray}
St  = & \frac{L_0}{v_0 t_0} &\qquad \mbox{Strouhal Zahl} \\
Re  = & \frac{\rho v_0 L_0}{\mu} &\qquad \mbox{Reynolds Zahl}
\label{eq:reynold_ferziger} \\
Fr  = & \frac{v_0}{\sqrt{L_0 g}} &\qquad \mbox{Froude Zahl} 
\end{eqnarray}
Meist bedeutet $v_0$ die Durchschnittsgeschwindigkeit und $L_0$ die Ausdehnung
des Systems.



%\newpage

{\bf Zusammenfassung Navier-Stokes Gleichungen}

Kontinuitätsgleichung:
\begin{equation}
\frac{\partial \rho }{\partial t} + \d (\rho \vec{v}) = 0
\end{equation}

Impulsgleichung mit dem Spannungstensor $T$ bzw. $S$ und der Körperkraft $f$:
\begin{equation}
\frac{\partial (\rho \vec{v})}{\partial t} + \d (\rho \vec{v}\otimes\vec{v})
= - \nabla p + \d S  + \rho \vec{f}
\end{equation}

Energiegleichung:
\begin{equation}
\frac{\partial (\rho E)}{\partial t} + \d (\rho \vec{v} H) = 
\d \left(S \vec{v} + \kappa \nabla T
\right) + q - \rho
\vec{v}\vec{g}
\end{equation}

{\bf Inkompressible Fluide}

Da die Dichte konstant ist, vereinfacht sich die Kontinuitätsgleichung
zu
\[
\d \vec{v} = 0
\]
Der Spannungstensor vereinfacht sich dann zu
\[
S = 2 \mu D = \mu\left[ (\nabla \otimes \vec{v})^T + (\nabla \otimes \vec{v})\right]
\]

{\bf Eulergleichungen}

Die Eulergleichungen beschreibt reibungslose Fluide, also mit der Viskosität $\mu=0$.

{\bf Kontinuitätsgleichung:} Keine Änderung

Impulsgleichung: Hier verschwindet der Term $S$ des
Spannungstensors, es verbleibt nur der Druckterm.
\begin{equation}
\frac{\partial (\rho \vec{v})}{\partial t} + \d (\rho \vec{v} \vec{v})
= \nabla p + \rho \vec{f}
\end{equation}


Energiegleichung: Auch hier verschwindet der Term $S$
\begin{equation}
\frac{\partial (\rho E)}{\partial t} + \d (\rho \vec{v} H) = - \d
W + q - \rho \vec{v}\vec{g}
\end{equation}
In Wikipedia unter Eulergleichung werden alle Quellterme gleich Null gesetzt
\begin{equation}
\frac{\partial (\rho E)}{\partial t} + \d (\rho \vec{v} H) = 0.
\end{equation}

Alle Gleichungen habe dieselbe Form (ohne Quell-Terme):
\[
\frac{\partial (\mbf{u})}{\partial t} + \d \mbf{f(u)} = S(u)
\]

Kontinuitätsgleichung: 
\begin{eqnarray*}
\mbf{u} &=& \rho\\
\mbf{f(u)} &=& \rho \vec{v}\\
S(\mbf{u}) &=& 0
\end{eqnarray*}

Impulsgleichung: 
\begin{eqnarray*}
\mbf{u} &=& \rho \vec{v}\\
\mbf{f(u)} &=& \rho \vec{v}\otimes\vec{v}\\
S(\mbf{u}) &=& \d (S - p I)  + \rho \vec{f}
\end{eqnarray*}

Energiegleichung: 
\begin{eqnarray*}
\mbf{u} &=& \rho E\\
\mbf{f(u)} &=& \rho \vec{v} H = 
\rho \vec{v} \left(E + \frac{p}{\rho} \right)\\
S(\mbf{u}) &=& \d \left(S \vec{v} - \vec{W}
\right) + q - \rho \vec{v}\vec{g} 
\end{eqnarray*}





%%% Local Variables: 
%%% mode: latex
%%% TeX-master: "cfd_multiflow"
%%% End: 

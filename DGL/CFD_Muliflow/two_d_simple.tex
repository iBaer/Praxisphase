\section{Zwei-Phasen-Strömung in 2 Dimensionen}

\subsection{Ein einfaches Modell in 2-D}

Es gibt zahlreiche Modelle und Simulationen, die bereits mit
Experimenten verglichen wurden. Die meisten nehmen jedoch an, dass die
Relativgeschwindigkeit zwischen den beiden Phasen zu vernachlässigen
ist. Hier soll die Relativgeschwindigkeit explizit berücksichtigt
werden. Eine einfache Erweiterung der Gleichungen
\ref{eq:dia_2009_k}\ref{eq:dia_2009_i} und \ref{eq:dia_2009_r} von
einer zu zwei Dimensionen ohne Quellterme kann über die Gleichungen
\ref{eq:masse_diff} und \ref{eq:impuls_diff} gewonnen werden. Die
lauten in Komponentenschreibweise für jede Phase

\begin{eqnarray}
\frac{\partial }{\partial t} (\alpha_i \rho_i) + 
\frac{\partial }{\partial x} (\alpha_i\rho_i v_{i,x}) +
\frac{\partial }{\partial y} (\alpha_i\rho_i v_{i,y}) &=& 0 \label{eq:rho_2}\\
%
\frac{\partial }{\partial t} (\alpha_i\rho_i v_{i,x}) +
\frac{\partial }{\partial x} (\alpha_i\rho_i v_{i,x}v_{i,x} + P) +
\frac{\partial }{\partial y} (\alpha_i\rho_i v_{i,x} v_{i,y}) &=& 0 \\
%
\frac{\partial }{\partial t} (\alpha_i\rho_i v_{i,y}) +
\frac{\partial }{\partial x} (\alpha_i\rho_i v_{i,x} v_{i,y}) +
\frac{\partial }{\partial y} (\alpha_i\rho_i v_{i,y}v_{i,y} + P)  &=& 0 
\end{eqnarray}
Die Rechnungen verlaufen ganz analog zu den Rechnungen aus Kapitel
\ref{sec:3pdfs_1d}. In den ersten beiden Gleichungen kommt nur der
Term mit der Ableitung nach $y$ hinzu, und die dritte Gleichung ist
symmetrisch zur zweiten aufgebaut. Aus diesem Grunde brauchen nur die
beiden zusätzlichen Terme der ersten beiden Gleichungen genauer
betrachtet werden. Addition von Gleichung \ref{eq:rho_2} für beide
Phasen ergibt eine Kontinuitätsgleichung wie gehabt.
\begin{equation}
\frac{\partial}{\partial t}\rho + \frac{\partial}{\partial x} \rho u_x
+ \frac{\partial}{\partial y} \rho u_y
=  0\label{eq:2d_simple_d}
\end{equation}
Bei der Impulsgleichung für die $x$-Komponente kommt folgender Term
hinzu
\[
\frac{\partial }{\partial y}(\phi_1+\phi_2):= \frac{\partial
}{\partial y} \left(\alpha_1\rho_1 v_{1,x} v_{1,y} + \alpha_2\rho_2
v_{2,x} v_{2,y}\right),
\]
bezeichnet mit $\phi_+$, der umgeformt werden
muss. Mit den analogen Gleichungen wie im 1-dimensionalen Fall:
\begin{eqnarray*}
u_{r,x} &=& v_{2,x}-v_{1,x}\\
u_x    &=& (1-c) v_{1,x} + c v_{2,x}
\end{eqnarray*}
folgt
\begin{eqnarray*}
v_{1,x} &=& u_{x}-c u_{r,x}\\
v_{2,x} &=& u_x + (1-c) u_{r,x}
\end{eqnarray*}
und damit
\begin{eqnarray*}
 \phi_+ = \alpha_1\rho_1 v_{1,x} v_{1,y} +
  \alpha_2\rho_2 v_{2,x} v_{2,y} &=& (1-c)\rho \left( u_{x}-c u_{r,x} \right)
\left( u_{y}-c u_{r,y} \right)\\
%
&& + c\rho \left( u_x + (1-c) u_{r,x} \right) \left( u_y + (1-c) u_{r,y} \right) \\
&=& \rho u_x u_y + \rho c(1-c) u_{r,x} u_{r,y} 
\end{eqnarray*}
mit $\alpha_1\rho_1 = (1-c)\rho$ bzw. $\alpha_2\rho_2 = c\rho$. Damit
ergeben sich die beiden Impulsgleichungen
%
\begin{eqnarray}
\frac{\partial}{\partial t} \rho u_x + \frac{\partial}{\partial x}
\left( \rho u_x^2 + \rho c (1-c) u_{r,x}^2 + P\right) +
\frac{\partial}{\partial y} \left(\rho u_x u_y + \rho c(1-c) u_{r,x} u_{r,y}\right)
 &=& 0 \label{eq:2d_simple_i_x}\\
%
\frac{\partial}{\partial t} \rho u_y + \frac{\partial}{\partial y} \left(
\rho u_y^2 + \rho c (1-c) u_{r,y}^2 + P\right)
 + \frac{\partial}{\partial x} \left(\rho u_x u_y + \rho c(1-c) u_{r,x} u_{r,y}\right) &=& 0\label{eq:2d_simple_i_y}
\end{eqnarray}
Die Herleitung der Gleichung für $u_{r,i}$ erfolgt analog zu
eindimensionalen Fall. Dort wurden zur Herleitung von Gleichung
\ref{eq:ur_1d} zuerst in der Gleichung für jede Phase einige
Umformungen vorgenommen und durch ${\alpha_i\rho_i}$ geteilt und dann
die Differenz beider Gleichungen betrachtet. Dementsprechend käme
jetzt nur der Term
\begin{eqnarray*}
\phi_- &=& v_{2,x} v_{2,y} - v_{1,x} v_{1,y}\\ 
&=& \left(u_x + (1-c) u_{r,x}\right) \left(u_y + (1-c) u_{r,y}\right)
- \left(u_{x}-c u_{r,x}\right)\left(u_{y}-c u_{r,y}\right)\\
&=&
u_x u_{r,y} + u_y u_{r,x} + (1-2c) u_{r,x} u_{r,y}
\end{eqnarray*}
hinzu. Damit lauten die beiden Gleichungen für die Relativgeschwindigkeit
\begin{eqnarray}
\frac{\partial}{\partial t} u_{r,x} + \frac{\partial}{\partial x}
\left( u_x u_{r,x} + \frac{1-2c}{2} u_{r,x}^2 + \Psi(P)\right) +
\frac{\partial}{\partial y}
\left(u_x u_{r,y} + u_y u_{r,x} + (1-2c)u_{r,x} u_{r,y} \right)
&=& 0.\qquad \label{eq:2d_simple_r_x}\\
%
\frac{\partial}{\partial t} u_{r,y} + \frac{\partial}{\partial y} \left(
u_y u_{r,y} + \frac{1-2c}{2} u_{r,y}^2 + \Psi(P)\right) 
+ \frac{\partial}{\partial x}
\left(u_x u_{r,y} + u_y u_{r,x} + (1-2c)u_{r,x} u_{r,y} \right)
&=& 0.\qquad  \label{eq:2d_simple_r_y}
\end{eqnarray}
mit $\Psi = \frac{\gamma}{\gamma-1} \frac{P}{\rho_2} -
\frac{P}{\rho_1}$ aus Gleichung \ref{eq:ur_Psi}. 

Die 5 Differentialgleichungen \ref{eq:2d_simple_d},
\ref{eq:2d_simple_i_x}, \ref{eq:2d_simple_i_y}, \ref{eq:2d_simple_r_x}
und \ref{eq:2d_simple_r_y} zusammen mit den gleichen Annahmen wie in
Kapitel \ref{sec:1d_simple}, also $\rho_1 = const$ und
\begin{equation}
P = K_2 \rho^\gamma_2\quad\mbox{und/oder}\quad P = K_\rho \rho^\gamma
\end{equation}
können ganz analog zum 1-dimensionalen Fall gelöst werden. Die
Ableitungen bzw. Flüsse in einer Richtung werden in zwei Richtungen
erweitert.


\subsection{Matrixschreibweise des Gleichungssystems in 2-d}

In erster Näherung kann auch dieselbe Gleichung für die CFL\_Bedingung
verwendet werden, falls mit analogen Anfangs- und Randbedingungen
gearbeitet wird. Besser ist es jedoch die Jacobi-Matrix wie für den
1-dimensionalen Fall in Kapitel \ref{sec:matrix_1d} beschrieben
herzuleiten.

\begin{equation}
  \frac{\partial U}{\partial t} + \frac{\partial F(U)}{\partial x}
  + \frac{\partial G(U)}{\partial y}=
0,\qquad t > 0, -\infty < x < \infty
\end{equation}
mit
\[
U = \left[\begin{array}{c}\rho \\ \rho u_x \\ \rho u_y \\ u_{r,x}
      \\ u_{r,y} \end{array}\right],
\quad 
%
F(U) = \left[\begin{array}{c}
    \rho u_x \\[2mm]
    \rho u_x^2 + \rho c (1-c) u_{r,x}^2 + P  \\[2mm]
    \rho u_x u_y + \rho c(1-c) u_{r,x} u_{r,y}\\[2mm]
    u_x u_{r,x} + \frac{1-2c}{2} u_{r,x}^2 + \Psi\\[2mm]
    u_x u_{r,y} + u_y u_{r,x} + (1-2c)u_{r,x} u_{r,y}\\[2mm]
  \end{array}\right]. 
\quad
G(U) = \left[\begin{array}{c}
    \rho u_y \\[2mm]
    \rho u_x u_y + \rho c(1-c) u_{r,x} u_{r,y}\\[2mm]
    \rho u_y^2 + \rho c (1-c) u_{r,y}^2 + P\\[2mm]
    u_x u_{r,y} + u_y u_{r,x} + (1-2c)u_{r,x} u_{r,y}\\[2mm]
    u_y u_{r,y} + \frac{1-2c}{2} u_{r,y}^2 + \Psi\\[2mm]
  \end{array}\right]. 
\]
bzw. in der Schreibweise der Variablen
\[
U = \left[\begin{array}{c} u_1 \\ u_2 \\ u_3 \\ u_{4}
      \\ u_{5} \end{array}\right],
\quad 
%
F(U) = \left[\begin{array}{c}
    u_2 \\[2mm]
    \frac{u_2^2}{u_1} + c (1-c) u_1 u_{4}^2 + P(u_1)  \\[2mm]
    \frac{u_2u_3}{u_1} + c(1-c) u_1 u_{4} u_{5}\\[2mm]
    \frac{u_2 u_{4}}{u_1}  + \frac{1-2c}{2} u_{4}^2 + \Psi(u_1)\\[2mm]
    \frac{u_2 u_{5}}{u_1}  + \frac{u_3}{u_1} u_{4} + (1-2c)u_{4} u_{5}\\[2mm]
  \end{array}\right]. 
\quad
G(U) = \left[\begin{array}{c}
    u_3 \\[2mm]
    \frac{u_2u_3}{u_1} + c(1-c) u_1 u_{4} u_{5}\\[2mm]
    \frac{u_3^2}{u_1} + c (1-c) u_1 u_{5}^2 + P(u_1) \\[2mm]
    \frac{u_2 u_{5}}{u_1}  + \frac{u_3 u_{4}}{u_1}  + (1-2c)u_{4} u_{5}\\[2mm]
    \frac{u_3 u_{5}}{u_1} + \frac{1-2c}{2} u_{5}^2 + \Psi(u_1)\\[2mm]
  \end{array}\right]. 
\]
Jetzt werden wieder die Ableitungen per Kettenregel ausgeführt, wobei
es eine Jacobi-Matrix für die Geschwindigkeiten in x-Richung und eine
für die Geschwindigkeiten in y-Richtung gibt.  Beide Matrizen sind in
wesentlichen Teilen identisch mit dem 1-dimensionalen Fall, d.h. bei
$J_x$ und bein $J_y$ kommen nur 2 neue Spalten bzw. Zeilen hinzu. Um
sich hier die Fallunterscheidungen zu ersparen, wird die Funktion
$\Psi$ und die Größe $P$ offen gelassen und müssen analog zum
1-dimensionalen Fall je nach Version ersetzt werden.

Für $J_x$:
\begin{eqnarray*}
%u_2
&& \partial_x (\frac{u_2^2}{u_1} + u_1 c (1-c) u_4^2 + P(u_1))\\
&& = \frac{2 u_2}{u_1} \partial_x u_2
                    - \frac{u_2^2}{u_1^2} \partial_x u_1
                    + c (1-c) u_4^2  \partial_x u_1
                    + u_1 c (1-c) 2 u_4  \partial_x u_4
                    + \partial_{u_1}P(u_1) \partial_x u_1\\
%u_3 NEU
&& \partial_x (\frac{u_2u_3}{u_1} + c(1-c) u_1 u_{4} u_{5})\\
&& = - \frac{u_2u_3}{u_1^2} \partial_x u_1 + \frac{u_3}{u_1} \partial_x u_2
     + \frac{u_2}{u_1} \partial_x u_3     
     + c (1-c) (u_4 u_5  \partial_x u_1 + u_1 u_5  \partial_x u_4
                + u_1 u_4  \partial_x u_5)
\\
%u_4                  
&& \partial_x (\frac{u_2 u_4}{u_1} + \frac{1-2c}{2}
                    u_4^2 + \Psi(u_1))\\
&& - \frac{u_2 u_4}{u_1^2}  \partial_x u_1
                    + \frac{1}{u_1} u_4 \partial_x u_2
                    + \frac{u_2}{u_1} \partial_x u_4 
                    + (1-2c) u_4 \partial_x u_4 
                    + \partial_{u_1}\Psi(u_1) \partial_x u_1\\
%u_5 NEU
&& \partial_x (\frac{u_2 u_{5}}{u_1} + \frac{u_3}{u_1}u_{4} + (1-2c)u_{4}u_{5})\\
&& - \frac{u_2 u_5}{u_1^2} \partial_x u_1 + \frac{u_5}{u_1} \partial_x u_2
   + \frac{u_2}{u_1} \partial_x u_5
   - \frac{u_3 u_4}{u_1^2} \partial_x u_1 + \frac{u_4}{u_1} \partial_x u_3
   + \frac{u_3}{u_1} \partial_x u_4
   + (1-2c)(u_{5}\partial_x u_{4} + u_{4}\partial_x u_{5})
\\
\end{eqnarray*}
Daraus lässt sich jetzt die Jacobi-Matrix für die $x$-Richtung ablesen.

\begin{equation}
J_x = \left(\begin{array}{ccccc}
%Zeile \rho
  0 & 1 & 0 & 0 & 0\\
%Zeile \rho u_x
 - \frac{u_2^2}{u_1^2} + c(1-c) u_4^2 + \partial_{u_1} P & \frac{2 u_2}{u_1} &  
 0 & c (1-c) 2 u_1 u_4 & 0 \\
%Zeile \rho u_y
 - \frac{u_2u_3}{u_1^2} +  c (1-c) u_4 u_5 & \frac{u_3}{u_1}  & \frac{u_2}{u_1} &
 c (1-c) u_1 u_5 & c (1-c)  u_1 u_4 \\
%Zeile u_rx
- \frac{u_2 u_4}{u_1^2} + \partial_{u_1} \Psi(u_1) &  \frac{u_4}{u_1}  & 0 &
 \frac{u_2}{u_1} + (1-2c) u_4 & 0\\
%Zeile u_ry
- \frac{u_2 u_5}{u_1^2} - \frac{u_3 u_4}{u_1^2} & \frac{u_5}{u_1} &
\frac{u_4}{u_1} & \frac{u_3}{u_1} + (1-2c) u_{5} & \frac{u_2}{u_1} + (1-2c)u_{4}
\end{array}\right)\qquad\label{eq:jacobi_x_2d_allgemein}
\end{equation}
Für die Jacobi-Matrix $J_y$ ist die 2. Zeile identisch mit der 3. von
$J_x$ und die 4. Zeile mit der 5. von $J_x$. In den anderen 3 Zeilen
wird $u_2$ durch $u_3$ und $u_4$ durch $u_5$ ersetzt, also
dementsprechend werden die Einträge in den Spalten der Zeilen 1, 2 und
4 vertauscht. Damit ergibt sich

\begin{equation}
J_y = \left(\begin{array}{ccccc}
%Zeile \rho
  0 & 0 & 1 & 0 & 0\\
%Zeile \rho u_x
 - \frac{u_2u_3}{u_1^2} +  c (1-c) u_4 u_5 & \frac{u_3}{u_1}  & \frac{u_2}{u_1} &
 c (1-c) u_1 u_5 & c (1-c)  u_1 u_4 \\
%Zeile \rho u_y
%umdrehen
 - \frac{u_3^2}{u_1^2} + c(1-c) u_5^2 + \partial_{u_1} P & 0 & \frac{2 u_3}{u_1} & 0 &  
 c (1-c) 2 u_1 u_5  \\
%Zeile u_rx
- \frac{u_2 u_5}{u_1^2} - \frac{u_3 u_4}{u_1^2} & \frac{u_5}{u_1} &
\frac{u_4}{u_1} & \frac{u_3}{u_1} + (1-2c) u_{5} & \frac{u_2}{u_1} + (1-2c)u_{4}\\
%Zeile u_ry
%umdrehen
- \frac{u_3 u_5}{u_1^2} + \partial_{u_1} \Psi(u_1) &  0 & \frac{u_5}{u_1}  & 0 &
 \frac{u_3}{u_1} + (1-2c) u_5 
\end{array}\right)\qquad\label{eq:jacobi_y_2d_allgemein}
\end{equation}


Für die noch zu berechnen Terme
\[
\partial_{\rho} P\quad\mbox{und}\quad \partial_{\rho} \Psi(\rho)
\]
ergbit sich das Gleiche wie in den 3 Varianten von Kapitel
\ref{sec:matrix_1d}.

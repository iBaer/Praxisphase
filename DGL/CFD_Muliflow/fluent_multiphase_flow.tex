\section{Multiphase Flow - implementiert in Fluent}

{\bf Fluent Handbuch, Kapitel 17 und Schulungsfolien}

In Fluent 6.3 sind vier Modelle für Mehrphasenströmungen
implementiert, die sich im wesentlichen dadurch unterscheiden, in wie
weit Oberflächen berücksichtigt werden und wie große die Anteile von
jeder Phase sind. Die Modelle sind
\begin{itemize}
\item Lagrangian Dispersed Phase Model (DPM): für Lagrange Teilchen,
  also feste Teilchen, Blasen und Tropfen
\item Volume of Fluid Model (VOF) zur Simulation von Grenzflächen
  zwischen den Phasen
\item Eulerian Model als Mittlung des VOF-Modells, geeignet zur
  Simulation von sich ausbreitenden oder auflösenden Phasen
\item Mixture Model, eine Vereinfachung des Euler Modells, anwendbar,
  wenn die Trägheit der dispersen Phase klein ist.
\end{itemize}
Betrachtet werden meist $n$ Komponenten, der Einfachheit halber werde
ich die Gleichungen auch zum besseren Vergleich mit der Arbeit von Dia
mit zwei Komponenten angeben. Alle Gleichungen müssen für die
einzelnen Komponenten sowie für die Wechselwirkung zwischen den
Komponenten betrachtet werden.

\subsection{\bf  Lagrangian Dispersed Phase Model (DPM)}

Das Modell wurde genauer bei Projekt der Bewegung von Fasern in
Strömungen untersucht (siehe Aufzeichnungen
fluent\_discrete\_phase\_model.tex) und wird auch als Euler-Lagrange
Ansatz bezeichnet. Es wird hier nicht weiter behandelt.

\subsection{\bf Volume of Fluid Model (VOF)}

Mit diesem Modell wurde aufsteigende größere Blasen oder das
Einfließen von Tinte untersucht. Es geht davon aus, dass die
Kontinuitätsgleichung für jede Phase einzeln ohne weitere Quellterme
gilt Es gibt eine klare Phasengrenze und nur ein Geschwindigkeitsfeld
$\vec{v}$ für beide Phasen, d.h. beide Phasen bewegen sich gleich
schnell.

In einem Volumen seien die Volumenanteile $\alpha_1$ und $\alpha_2$
der beiden Phasen und damit die Massenanteile $\alpha_1\rho_1$ und
$\alpha_2\rho_2$ und die mittlere Dichte $\rho_m = \alpha_1\rho_1 +
\alpha_2\rho_2$. Betrachtet man die beiden Kontinuitätsgleichungen für
dei beiden Phasen,
\begin{eqnarray*}
\frac{\partial \alpha_1 \rho_1 }{\partial t} + \d (\alpha_1\rho_1 \vec{v}) 
&=& 0\\
\frac{\partial \alpha_2 \rho_2 }{\partial t} + \d (\alpha_2\rho_2 \vec{v}) 
&=& 0
\end{eqnarray*}
dann verwendet das VOF-Modell nur die addierten Gleichungen für die
mittlere Dichte
\begin{eqnarray*}
\frac{\partial \rho_m }{\partial t} + \d (\rho_m \vec{v}) 
&=& 0.
\end{eqnarray*}
In die Impulsgleichungen gehen die Kräfte für die Oberflächen zwischen
den Phasen ein und die Bewegung der Oberflächen wird bestimmt. Dazu
wird nur eine Impulsgleichung für beide Phasen bestimmt:
\[
\frac{\partial \rho \vec{v}}{\partial t} + \d (\rho
\vec{v}\otimes\vec{v}) = - \nabla p + \d [\mu ( \nabla \otimes
\vec{v})^T + (\nabla \otimes \vec{v}))] + \rho \vec{g} + \vec{F}
\]
Der Kraftterm beinhaltet die Oberflächenspannung $T_\sigma$ in der
Grenzschicht zwischen den Phasen.


{\bf Achtung:} der Termin $-\frac{2}{3} (\d \vec{v}_k)\; I $ im
Spannungstensor tritt im Handbuch {\bf nicht} auf (Kapitel 17.3.6. des
Handbuchs), wird also vernachlässigt, obwohl eine Phase kompressible
sein darf!

Die gemeinsame Energiegleichung für beide Phasen lautet
\begin{equation}
\frac{\partial (\rho E)}{\partial t} + \d (\rho \vec{v} E + p) = \d
(\kappa \nabla T) + q - \rho \vec{v}\vec{g} 
\end{equation}
Hier wird der gesamte Spannungstensor weggelassen (Kapitel 17.3.7. des
Handbuchs)!

Das ist das sogenannte ``one fluid model''. Es existiert auch ein VOF
``multi fluid model'', bei dem die Kontinuitätsgleichung für jede
Phase mit einem Austauschterm zwischen den Phasen berücksichtigt wird
sowie 2 Impulsgleichungen, ähnlich dem Euler-Modell.


\subsection{\bf Eulerian Model} \label{subsec:Euler}

Das gibt beiden Phasen eine unterschiedliche Geschwindigkeit und lässt
einen Änderung der Masse von einer Phase in die andere zu (ist im
Fluent-Handbuch auch für das VOF-Modell der Fall). Damit bekommt die
Kontinuitätsgleichungen einen Quellterm
\begin{eqnarray}
\frac{\partial \alpha_1 \rho_1 }{\partial t} + \d (\alpha_1\rho_1 \vec{v}_1) 
&=& S_{\alpha_1} + \dot{m}_{21}\\ \nonumber
\frac{\partial \alpha_2 \rho_2 }{\partial t} + \d (\alpha_2\rho_2 \vec{v}_2) 
&=&  S_{\alpha_2} + \dot{m}_{12} \label{eq:eulerKonti}
\end{eqnarray}
Die Quelle $S$ ist im Normalfall 0. Bei der Impulsgleichung werden neben
dem Spannungstensor und der Gravitationskraft Kräfte zwischen den
Phasen berücksichtigt. Nach den Folien lautet die Impulsgleichung
\begin{eqnarray}
\frac{\partial \alpha_1 \rho_1 \vec{v}_1}{\partial t} + \d
(\alpha_1\rho_1 \vec{v}_1\otimes\vec{v}_1) &=& - \alpha_1 \nabla p +
\alpha_1\rho_1 \vec{g} + \d (\alpha_1 S) + K_{12} (\vec{v}_2 -
\vec{v}_1) + F_{12}\\ \nonumber
\frac{\partial \alpha_1 \rho_2 \vec{v}_2}{\partial t} + \d
(\alpha_2\rho_2 \vec{v}_2\otimes\vec{v}_2) &=& - \alpha_2 \nabla p +
\alpha_1\rho_2 \vec{g} + \d (\alpha_2 S) + K_{12} (\vec{v}_1 -
\vec{v}_2) + F_{12} \label{eq:eulerImp}
\end{eqnarray}
Der Anteil des Spannungstensors proportional zur Viskosität $S$ ist
analog aufgebaut wie der Term in der Einphasenströmung, ist aber nicht
genau bekannt. Der wichtigste Termin ist die Zugkraft oder der
Widerstandsterm $K_{12}$, der ebenfalls nur angenähert werden kann.

Die Energiegleichung wird im Fluent-Tutorial für $H$ und nicht für $E$
aufgestellt. Mit $\rho H = \rho E+p$, taucht die zeitlich Ableitung
des Drucks auf.

\begin{eqnarray}
\frac{\partial (\alpha_1 \rho H_1)}{\partial t} + \d (\alpha_1\rho_1
\vec{v}_1 H_1) &=& - \frac{\partial (\alpha_1 p)}{\partial t} +
\d \left(\alpha_1 S \vec{v}_1 - \vec{W}_1 \right) + q_1 - \rho_1
\vec{v}_1\vec{g} \nonumber \\
& &  + Q_{12} (T_2-T_1) - \dot{m}_{12} h_1 \nonumber \\
\frac{\partial (\alpha_2 \rho H_2)}{\partial t} + \d (\alpha_2\rho_2
\vec{v}_2 H_2) &=& - \frac{\partial (\alpha_2 p)}{\partial t} +
\d \left(\alpha_2 S \vec{v}_2 - \vec{W}_2 \right) + q_2 - \rho_2
\vec{v}_2\vec{g} \nonumber \\
& &  + Q_{12} (T_1-T_2) + \dot{m}_{21} h_2 \label{eq:eulerE}
\end{eqnarray}

{\bf Achtung:} Auf dem Fluent-Folien steht beim Term $\d \alpha_1
S $ ein $\nabla \vec{v}_1$, das dürfte aber eigentlich nicht sein,
kommt schon mit den Einheiten nicht hin. Hingegen im Handbuch steht
$\alpha_1 S \nabla \vec{v}_1$, also das $\nabla$ steht nur vor $v$.

Hinzu sind die beiden Terme $Q_{12} (T_2-T_1)$ und $\dot{m}_{12} h_1$
gekommen, die die Energie für den Wärmeaustausch und potentielle
Energie beim Massenaustausch berücksichtigen.



\subsection{\bf Mixture Model}

Beim Mixture Model wird analog zu zwei Teilchen in
Schwerpunktskoordinaten bzw. in über die Dichten gemittelte Koordinaten
übergegangen. Bei zwei Teilchen ist die Geschwindigkeit des
Schwerpunkts gegeben durch
\[
\vec{v}_m = \frac{m_1\vec{v}_1+m_2\vec{v}_2}{m_1+m_2}
\]
Damit ergibt sich analog zur Schwerpunktsbewegung eine über die Massen
gemittelte Geschwindigkeit
\[
\vec{v}_m = \frac{\alpha_1\rho_1 \vec{v}_1 + \alpha_2\rho_2 \vec{v}_2}{\rho_m}
\]

{\bf Gültigkeit}

Annahmen, die Einfluss auf die Gleichungen haben sind:
\begin{itemize}
\itemsep 0pt
\item Nur eine Phase kann ein kompressibles ideales Gas sein.
\item Es ist keine nichtviskose oder reibungsfreie Strömung erlaubt.
\item Es erlaubt im Gegensatz zum VOF-Modell unterschiedliche
  Strömungungsgeschwindigkeiten.
\end{itemize}
Daneben gibt es noch zahlreiche andere Einschränkungen (siehe Handbuch
Kapitel 17.4.2)


{\bf Kontinuitätsgleichung:}

Werden die beiden Euler-Gleichungen \ref{eq:eulerKonti} addiert ergibt
sich für $S_{\alpha_1}=S_{\alpha} = 0$ und $ \dot{m}_{12} = -
\dot{m}_{21}$ eine Kontinuitätsgleichung für die gemittelte Dichte
$\vec{v}_m$
\begin{equation}
\frac{\partial \rho_m }{\partial t} + \d (\rho_m \vec{v}_m) =
0 \label{eq:fluent-mixed-konti}
\end{equation}
und eine Gleichung für die Anteile in der jeweiligen Phase, bei Fluent
als {\it Volume Fraction Equation} bezeichnet, wenn die erste
Gleichung von der 2. subtrahiert wird. Diese wird mit der
Relativgeschwindigkeit bzw. Driftgeschwindigkeit $\vec{v_{dr,k}}$
zwischen den Komponenten formuliert
\[
\vec{v}_{dr,k} = \vec{v}_{k}-\vec{v}_{m}.
\]
$k$ bezeichnet die betrachtete Phase. Damit ergibt sich
\[
\frac{\partial \alpha_2 \rho_2 }{\partial t} + \d (\alpha_2\rho_2
\vec{v}_2) = -\d (\alpha_2\rho_2\vec{v_{dr,2}}) + (\dot{m}_{12} -
\dot{m}_{21})
\]

{\bf Impulsgleichung:}

Als Impulsgleichung wird eine ähnliche Gleichung wie im VOF-Modell
angegeben: 
\begin{equation}
\frac{\partial \rho_m \vec{v}_m}{\partial t} + \d \left(\rho_m
\vec{v}_m\otimes\vec{v}_m\right) = - \nabla p + \d [\mu_m (( \nabla
  \otimes \vec{v}_m)^T + (\nabla \otimes \vec{v}_m))] + \rho_m \vec{g}
+ \vec{F} + \d \left(\sum_{k=1}^2\alpha_k\rho_k
\vec{v}_{dr,k}\otimes\vec{v}_{dr,k}\right)\label{eq:fluent-mixed-impuls}
\end{equation}
mit $\mu_m = \alpha_1 \mu_1 + \alpha_2 \mu_2$.


Mögliche Herleitung: Ausgangspunkt sind Impulsgleichungen des
Euler-Modells, Gleichung \ref{eq:eulerImp}, wobei die Kräfte zwischen
den Phasen zu $F = K_{12} (\vec{v}_2 - \vec{v}_1) + F_{12}$
zusammengefasst werden. Dann folgt für jede Phase $k$ für sich mit
Einsetzung des Spannungstensors
\[
\frac{\partial \alpha_k \rho_k \vec{v}_k}{\partial t} + \d
(\alpha_k\rho_k \vec{v}_k\otimes\vec{v}_k) = - \nabla \alpha_k p_k +
\alpha_k\rho_k \vec{g} + \d (\alpha_k 
\mu_k ( \nabla \otimes \vec{v}_k)^T + (\nabla \otimes \vec{v}_k) -
\frac{2}{3} (\d \vec{v}_k)\; I ) + \vec{F}
\]

Werden die beiden Gleichungen addiert, ergibt sich
\begin{eqnarray*}
\frac{\partial \rho_m \vec{v}_m}{\partial t} + \d
\left(\sum_{k=1}^2\alpha_k\rho_k \vec{v}_k\otimes\vec{v}_k\right)
&=& -
\nabla p + \d \sum_{k=1}^2 \left(\alpha_k \mu_k (
\nabla \otimes \vec{v}_k)^T + (\nabla \otimes \vec{v}_k) - \frac{2}{3}
(\d \vec{v}_k\; I )\right)\\
&&
\rho_m \vec{g} + \vec{F}
\end{eqnarray*}
mit $\alpha_1 p_1 + \alpha_2 p_2 = p$ und damit komme ich leider nicht
auf die Gleichung aus dem Fluent-Handbuch!  Es fehlt wieder der
$\nabla^2 \vec{v}_k$ und beim Term mit den Driftgeschwindigkeit
bekomme ich etwas anderes heraus:
\begin{eqnarray*}
&& \rho_m \vec{v}_m\otimes\vec{v}_m - \sum_{k=1}^2\alpha_k\rho_k
\vec{v}_{dr,k}\otimes\vec{v}_{dr,k} \\
&& =(\alpha_1\rho_1+\alpha_2\rho_2)\vec{v}_m\otimes\vec{v}_m -
\alpha_1\rho_1 (\vec{v}_1-\vec{v}_m)\otimes( \vec{v}_1-\vec{v}_m) -
\alpha_2\rho_2 (\vec{v}_2-\vec{v}_m)\otimes( \vec{v}_2-\vec{v}_m) \\
&& = \alpha_1\rho_1 [(\vec{v}_m\otimes\vec{v}_1) +
                     (\vec{v}_1\otimes\vec{v}_m) -
                     (\vec{v}_1\otimes\vec{v}_1)] +
     \alpha_2\rho_2 [(\vec{v}_m\otimes\vec{v}_2) +
                     (\vec{v}_2\otimes\vec{v}_m) -
                     (\vec{v}_2\otimes\vec{v}_2)]\\
&& \neq
\alpha_1\rho_1 \vec{v}_1\otimes\vec{v}_1 +
\alpha_2\rho_2 \vec{v}_2\otimes\vec{v}_2
\end{eqnarray*}

{\bf Achtung:} Nur für $\vec{v}_1\approx\vec{v}_2\approx\vec{v}_m$
stimmt der Ausdruck.

{\bf Achtung:} Der Anteil $\nabla^2 \vec{v}_k$ des Spannungstensors in
der Impulsgleichung für eine Phase fehlt, die nach Voraussetzung auch
kompressible sein kann


{\bf Energiegleichung}

Es wird für beide Phasen eine gemeinsame Energiegleichung aufgestellt. 
\begin{equation}
\frac{\partial}{\partial t}\sum_{k=1}^2 (\alpha_k \rho_k E_k)
+ \d (\sum_{k=1}^2 \alpha_k \rho_k \vec{v}_k H_k) =
\d (\kappa \nabla T_{eff}) + S_E\label{eq:fluent-mixed-energie}
\end{equation}
$S_E$ beinhaltet weitere Wärmequellen.  Das ist die Summe der
Eulergleichungen für die Energie von beiden Phasen (\ref{eq:eulerE})
mit $E=H-\rho p$, dem Tensor $S=0$ und ohne Gravitation. 

{\bf Achtung:} Energiegleichung ohne Spannungstensor widerspricht der
Voraussetzung, dass die Strömung Reibung haben muss, scheint für die
Energiegleichung aber zu vernachlässigen zu sein.



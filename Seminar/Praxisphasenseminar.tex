% -*-mode:LaTeX; coding: utf-8;-*-

% Wahl des Ausgabemodus
\documentclass{beamer}           % Präsentation (Farbe mit \pause)
%\documentclass[handout]{beamer} % Handoutversion (s/w und 4up ohne \pause)
%\documentclass[trans]{beamer}   % wie Präsentation, aber ohne \pause

\usepackage[ngerman]{babel}

% Anpassen ans eingestellte Encoding
%\usepackage[latin1]{inputenc}
\usepackage[utf8]{inputenc}

\usepackage{graphicx}
\usepackage{epsfig}

\usepackage[normalem]{ulem}


\usetheme{HN}

% für anderes Logo als fb03:
\setHNlogo{imh}

% Navigationselemente verstecken
\beamertemplatenavigationsymbolsempty

% wenn andere Fußzeile gewünscht, auskommentieren und anpassen:
% footline
%
%\setbeamertemplate{footline}[text line]{
%   \begin{beamercolorbox}[wd=0.98\paperwidth]
%      \insertshortauthor:~\insertshorttitle.~-\insertframenumber-
%      %\inserttotalframenumber
%      \hfill \vskip 0.1cm
%   \end{beamercolorbox}
%}
                             


%------------------------------------------------------------------------
% Titelseite
\title[Praxisphase]{Praxisphase}

\author[P. Bähr]{Pascal Bähr}
\institute{Hochschule Niederhein - Fachbereich Elektrotechnik \& Informatik}
\date{28. April 2016}

\begin{document}
\frame[plain]{\titlepage}

\begin{frame}{Übersicht}
	\begin{itemize}
		\item Arbeitgeber
		\item Arbeitsorganisation
		\item Vereinbarungen
		\item Aufgabe
		\item Fachliche Aspekte
		\item Gewonnene Erfahrungen
		\item Rest der Praxisphase
		\item Zusammenfassung
	\end{itemize}
\end{frame}

%-----------------------------------------------------------
% Arbeitgeber
\disableLogo
\begin{frame}{Arbeitgeber}{IMH}
	\begin{center}
		\includegraphics[scale=0.4]{beamerthemeHN/imh.eps}
	\end{center}
	\begin{block}{Betreuer}
		Prof. Dr. Peer Ueberholz
	\end{block}
\end{frame}
\enableLogo

\begin{frame}{Arbeitgeber}{IMH}
	\begin{block}{IMH - Institut für Modellbildung und Hochleistungsrechnen}
		\pause
		Leistungsspektrum \& Forschung
		
		\begin{itemize}
			% Edit: Eventuell alle Items gleichzeitig für schnellen Überblick
			\item<2-> Maßgeschneiderte Numerische Strömungssimulation (Computational Fluid Dynamics)
			
			\item<3-> Paralleles und verteiltes Rechnen
			
			\item<4-> Design- und Prozess-Optimierung und stochastische Analyse
		\end{itemize}
	\end{block}	
\end{frame}

\begin{frame}{Arbeitgeber}{IMH}
	\begin{block}{Labor}

		\begin{itemize}
			\item Hochschule Niederrhein
			
			\item Reinarzstr. 49
			
			\item 47805 Krefeld
			
			\item Raum: B313		
		\end{itemize}
	\end{block}
	\pause
	\begin{block}{Ausstattung}
		
		\begin{itemize}
			% Edit: Eventuell alle Items gleichzeitig für schnellen Überblick
			\item<2-> Workstations mit bis zu \alert{96GB RAM} und schneller Grafikkarte
			
			\item<3-> Mehrere Cluster mit bis zu \alert{2048GB RAM} und bis zu \alert{512 Rechenkernen}
			
		\end{itemize}
	\end{block}
\end{frame}


%-----------------------------------------------------------
% Arbeitsorganisation
\begin{frame}{Arbeitsorganisation}
	\begin{itemize}
		\item Selbstständig
		\pause
		\item Eigenverantwortlich
		\pause
		\item Im regelmäßigen Austausch mit den Betreuern
		\pause
		\item Projektorientiert
		\pause
		\item Ergebnisorientiert
	\end{itemize}
\end{frame}

%-----------------------------------------------------------
% Vereinbarte Tätigkeiten
\begin{frame}{Vereinbarungen}
	Mitarbeit an einem aktuellen Forschungsthema, durch
	\begin{itemize}
		\item Einarbeitung
		\pause
		\item \alt<2>{Entwicklung oder Weiterentwicklung}{\sout{Entwicklung oder} \alert{Weiterentwicklung}\pause}
		\pause
		\begin{itemize}
			\item Refactoring
			\item Weiterentwicklung
			\begin{itemize}
				\item \alert{Bachelorarbeit?}
			\end{itemize}
		\end{itemize}
	\end{itemize}
\end{frame}

\begin{frame}{Aufgabe}
	Gegeben:
	\begin{itemize}
		\item Ein C++-Programm zur Simulation von Mehrphasenströmungen
		\item Die dazugehörige Bachelorarbeit
		\item Ausreichend Literatur: englische Bücher und Paper
	\end{itemize}
	\pause
	TODO:
	\begin{itemize}
		\item Einarbeitung in
		\begin{itemize}
			\item (partielle) Differentialgleichungen
			\item Grundlagen Mehrphasenströmung
			\item Numerische Lösungsverfahren
		\end{itemize}
		\pause
		\item \alert{Refactoring des Programms}
	\end{itemize}
\end{frame}

\begin{frame}{Fachliche Aspekte}
	\begin{block}{Mehrphasenströmung}
		\begin{itemize}
			\item Teilgebiet der Strömungsmechanik
			\pause
			\item Betrachtung von Strömungen mehrerer Stoffe
			\begin{itemize}
				\item Liquide
				\item Gase
			\end{itemize}
			\pause
			\item Herangehensweise
			\begin{itemize}
				\item Simulierende Gebiet wird auf Zellen aufgeteilt (Vgl. Pixel)
				\pause
				\item In jeder Zelle befindet sich ein Gas-Flüssigkeits-Gemisch
				\pause
				\item Fluss, wenn Flüssigkeit oder Gas sich in eine andere Zelle bewegt
				\begin{itemize}
					\item den es schrittweise zu berechnen gilt
				\end{itemize}
			\end{itemize}
			\pause
			\item Beispiele
			\begin{itemize}
				\item Vulkanaktivitäten
				\item Flüsse und Aktivitäten eines Motors
			\end{itemize}
		\end{itemize}
	\end{block}	
\end{frame}

\begin{frame}{Fachliche Aspekte}
	\begin{block}{Differentialgleichungen}
		\begin{itemize}
			\item Funktion ist nur implizit gegeben durch
			\pause
			\begin{itemize}
				\item eine Gleichung für die Ableitung
				\pause
				\item eine Anfangsbedingung
				\pause
			\end{itemize}
			\item Beispiel:
		\end{itemize}
		\begin{columns}
			\begin{column}{.1\textwidth}
				Ableitung:
			\end{column}
			\begin{column}{.3\textwidth}
				$y'(t) = 6t^2$
			\end{column}
		\end{columns}
		\begin{columns}
			\begin{column}{.1\textwidth}
				Anfangsbedingung:
			\end{column}
			\begin{column}{.3\textwidth}
				$y'(0) = 2$
			\end{column}
		\end{columns}
		\pause
		\begin{columns}
			\begin{column}{.1\textwidth}
				\alt<5>{Gesucht:}{Lösung:\pause}
			\end{column}
			\begin{column}{.3\textwidth}
				\alt<5>{$y(t) = ?$}{$y(t) = 2t^3 + 2$}
			\end{column}
		\end{columns}
		\pause
		\begin{itemize}
			\item Numerische Lösungen sind Approximationen
		\end{itemize}
	\end{block}	
\end{frame}

\begin{frame}{Fachliche Aspekte}
	\begin{block}{Partielle Differentialgleichungen}
		\begin{itemize}
			\item Meistens Orts- und zeitabhängige Funktionen gesucht
			\pause
			\item Partielle Ableitungen gegeben
			\pause
			\item Beispiel:
		\end{itemize}
		\hspace{2em}Laplace-Operator: $\Delta f =
		\frac{\partial^2 f}{\partial x^2} +
		\frac{\partial^2 f}{\partial y^2}$\\
		\hspace{4em}- bekannt aus GRA
	\end{block}	
\end{frame}

%-----------------------------------------------------------
% Durchgeführte Tätigkeiten
\begin{frame}{Fachliche Aspekte}
	\begin{block}{Refactoring - Vorgehensweise}
		\begin{itemize}
			\item Analyse des vorliegenden Programms
			\pause
			\item Schreiben eines Tests
			\pause
			\item Testen von Varianten zum Parsen von Gleichungen auf Effizienz
			\pause
			\item Plan mit groben Änderungen erstellen
			\pause
			\item \alert{Plan umsetzen}
		\end{itemize}
	\end{block}	
\end{frame}

%-----------------------------------------------------------
% normale Seite
\begin{frame}{Gewonnene Erfahrungen}
	\begin{itemize}
		\item Berufsorientierter Tagesablauf
		\pause
		\item Einblicke in:
		\begin{itemize}
			\item Wissenschaftliches Arbeiten
			\item Forschung
			\item Software Refactoring
		\end{itemize}
		\pause
		\item Affinität zum wissenschaftlichen Arbeiten
		\begin{itemize}
			\item mathematische \&
			\item numerische Themen
		\end{itemize}
	\end{itemize}
\end{frame}

\begin{frame}{Rest der Praxisphase?}
	\begin{itemize}
		\item Bestätigen, dass mein Weg, weiterhin wissenschaftlich zu Arbeiten, der richtige ist
		\pause
		\item Kann ich in dem zeitlichen Rahmen eine Lösung erarbeiten, die den Anforderungen genügt?
		\pause
		\item Bessere Einschätzung der Fähigkeiten \& Effizienz
	\end{itemize}
\end{frame}

\begin{frame}{Zusammenfassung}
	Die Praxisphase im IMH hat mich bisher vor neue Aufgaben gestellt und mir erste Einblicke in einen geregelten Arbeitsalltag gegeben.\\
	\pause
	\vspace{2em}
	Das Thema sagt mir aufgrund meiner Affinität zu numerischen und mathematischen Themen zu.\\
	\pause
	\vspace{2em}
	Bisher denke ich, die Aufgaben in einem angemessene Zeitfenster bewerkstelligt zu haben. Auch der direkte Austausch mit meinem Betreuer bestätigt mir dies.
\end{frame}

\end{document}

% -*-mode:LaTeX; coding: utf-8;-*-

% Wahl des Ausgabemodus
\documentclass{beamer}           % Präsentation (Farbe mit \pause)
%\documentclass[handout]{beamer} % Handoutversion (s/w und 4up ohne \pause)
%\documentclass[trans]{beamer}   % wie Präsentation, aber ohne \pause

\usepackage[ngerman]{babel}

% Anpassen ans eingestellte Encoding
%\usepackage[latin1]{inputenc}
\usepackage[utf8]{inputenc}


\usetheme{HN}

% für anderes Logo als fb03:
\setHNlogo{fb03}

% Navigationselemente verstecken
\beamertemplatenavigationsymbolsempty

% wenn andere Fußzeile gewünscht, auskommentieren und anpassen:
% footline
%
%\setbeamertemplate{footline}[text line]{
%   \begin{beamercolorbox}[wd=0.98\paperwidth]
%      \insertshortauthor:~\insertshorttitle.~-\insertframenumber-
%      %\inserttotalframenumber
%      \hfill \vskip 0.1cm
%   \end{beamercolorbox}
%}
                             


%------------------------------------------------------------------------
% Titelseite
\title[Kurztitel]{Voller Titel}

\author[Kurzautor]{Voller Autor}
\institute{Hochschule Niederhein - Fachbereich Elektrotechnik \& Informatik}
\date{3. Mai 2011}

\begin{document}
\frame[plain]{\titlepage}



%-----------------------------------------------------------
% normale Seite
\begin{frame}{Blabla}  
Blabla

\begin{block}{Blabla}
Noch mehr Blabla
\end{block}

Beispiel für {\em enumerate}:
\begin{enumerate}
\item erster Punkt
\item zweiter Punkt
\end{enumerate}

\vspace{1ex}
Der {\em \textbackslash pause} Befehl hält an, aber
nur im Modus {\em beamer},\\
nicht in den Modi {\em trans} und {\em handout}
\pause

\vspace{1ex}
Beispiel für {\em itemize}:
\begin{itemize}
\item erster Punkt
\item zweiter Punkt
\end{itemize}

\vspace{1ex}
Beispiel für {\em alert}:\\
Dieser Text ist \alert{farblich hervorgehoben}

\end{frame}


%-----------------------------------------------------------
% Seite ohne Logo
\disableLogo
\begin{frame}{Seite ohne Logo}
   Das Logo kann unterdrückt werden mit {\tt \textbackslash disableLogo}
   vor dem Frame.

   Anschließend kann es wieder mit {\tt \textbackslash enableLogo}
   aktiviert werden.

   \vspace{1ex}
   Das kann z.B. sinnvoll sein, wenn das Logo sich mit einer Abbildung
   überschneiden würde.
\end{frame}
\enableLogo


%-----------------------------------------------------------
% Seite ohne Logo und ohne Kopf- und Fußzeile
\disableLogo
\begin{frame}[plain]{Seite ohne ...}
   ... Logo, Kopfzeile, Fußzeile mittels
   \begin{itemize}
   \item {\tt \textbackslash disableLogo} vor
   \item {\tt \textbackslash begin\{frame\}[plain]}
   \end{itemize}
\end{frame}
\enableLogo


%-----------------------------------------------------------
% Seite mit Logo, aber ohne Kopf- und Fußzeile
\begin{frame}[plain]{Seite mit Logo, aber ohne ...}
   ... Kopfzeile, Fußzeile mit
   \begin{quote}
   {\tt \textbackslash begin\{frame\}[plain]}
   \end{quote}

\end{frame}




%-----------------------------------------------------------
% normale Seite mit Spalten
\begin{frame}{Spaltenseite}

  Spalten kann man mit {\tt \textbackslash begin\{columns\}} usw. erzeugen:
  \vspace{2ex}
   \begin{columns}
      \begin{column}{.4\textwidth}
         Dies ist der Text in der ersten Spalte.
      \end{column}
      \begin{column}{.4\textwidth}
         Dies ist der Text in der \emph{zweiten} Spalte.
      \end{column}
   \end{columns}
   \vspace{4ex}
   Bemerkung: ginge natürlich auch mit der {\em tabular}-Umgebung.
   

\end{frame}


\end{document}
